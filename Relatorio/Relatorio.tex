\documentclass[a4paper]{article}
% Pacotes necessários
\usepackage[portuguese]{babel}
\usepackage[backend=biber, style=apa, citestyle=apa, language=portuguese]{biblatex}
\usepackage{csquotes}
\addbibresource{Recursos/referencias.bib}

\usepackage{amsmath}
\usepackage{graphicx}
\usepackage{subcaption}
\usepackage{setspace}
\usepackage{siunitx} % Required for alignment
\sisetup{
  round-mode          = places, % Rounds numbers
  round-precision     = 2, % to 2 places
}
\usepackage{enumerate}
\usepackage{enumitem}
\usepackage{amsmath}
\usepackage{karnaugh-map}
\usepackage[section]{placeins}
\usepackage{geometry}
\usepackage{amssymb}
\usepackage{titling}
\usepackage[T1]{fontenc}
\usepackage{float}
\usepackage[hidelinks]{hyperref}
\usepackage{xcolor}
\usepackage{indentfirst}
\usepackage{array}
\usepackage{soul}
\usepackage{afterpage}
\newcolumntype{P}[1]{>{\centering\arraybackslash}p{#1}}
\onehalfspacing


% Comando para criar uma página vazia
\newcommand\myemptypage{
    \null
    \thispagestyle{empty}
    \addtocounter{page}{-1}
    \newpage
}

% Página de título principal
\newcommand{\firsttitlepage}{
    \begin{titlepage}
        \centering
        \vspace*{1cm}
        
        % Logos superior
        \begin{figure}[h!]
            \centering
            \includegraphics[width=6cm]{Recursos/LOGO_IPB} % Substitua pelo caminho da imagem
            \vspace{0.5cm}
        \end{figure}

        % Informações da instituição
        \large\textbf{INSTITUTO POLITÉCNICO DE BEJA} \\
        \large\textbf{Escola Superior de Tecnologia e Gestão} \\
        \large\textbf{Mestrado em Engenharia de Segurança Informática} \\
        \large\textbf{Direito na Segurança Informática e no Cibercrime} \\
        
        \vspace{2cm}
        
        % Título do projeto
        {\Huge \textbf{Caso 1}} \\
        
        \vspace{1.5cm}
        
        % Autores
        \large Paulo António Tavares Abade - 23919 \\
        
        \vfill
        
        % Logo inferior
        \begin{figure}[h!]
            \centering
            \includegraphics[width=6cm]{Recursos/IPBejaESTIG.jpg} % Substitua pelo caminho da imagem
        \end{figure}
        
        \vspace{1cm}
        
        % Local e data
        {\large Beja, abril de 2025}
    \end{titlepage}
}


\begin{document}


\pagenumbering{gobble} % Oculta numeração da página

% Primeira página de título
\firsttitlepage

%\secondtitlepage


% Abstract
%\section*{\LARGE\textbf{\textit{Resumo}}}

%Resposta ao Caso 1 da disciplina de Regulação Informática, onde foi necessário analisar informação...


\vspace{1cm}
% Keywords
%\textbf{Keywords:} rgdp
\newpage
%--------------------------------------------------------------------------------------------------------------------------------------

%\section*{\LARGE\textbf{\textit{Abstract}}}

%Response to Case 1 of the discipline of Computer Regulation, where it was necessary to analyze information...




\vspace{1cm}
% Keywords
%\textbf{Keywords:} rgdp
%\renewcommand{\contentsname}{Índice}       % Título do sumário
%\renewcommand{\listfigurename}{Índice de Figuras} % Título da lista de figuras

% Início do conteúdo do relatório
\newpage
\doublespacing
%\tableofcontents
%\listoffigures
\doublespacing

\newpage
\pagenumbering{arabic}

\section{Datel VS. Sony}\label{intro}
Em 2012, a \textit{Sony} iniciou uma tentativa de processo judicial contra a \textit{Datel}, uma empresa conhecida por fabricar diversos acessórios 
para consolas de jogos, como a Playstation Portable (\textit{PSP}), sendo esta o alvo do processo em questão. A \textit{Datel} produzia e 
vendia dispositivos de batota que permitiam aos jogadores modificar jogos, desbloquear conteúdos que normalmente estariam inacessíveis, no caso 
o Action Replay. Sendo assim, a \textit{Sony} alega que com isto a \textit{Datel} estava a infringir os seus direitos de propriedade intelectual, nomeadamente o código 
do jogo, encaixando isto no Artigo 4º, alínea b) da Diretiva 2009/24/CE (\cite{diretiva-2009-24}) onde é referido que "A tradução, adaptação, ajustamentos ou outras modificações
do programa e a reprodução dos respectivos resultados, sem prejuízo dos direitos de autor da pessoa que altere o programa", onde seria necessária autorização 
prévia da \textit{Sony} para a produção e venda destes dispositivos. A \textit{Sony} ainda argumentou para considerar o código como uma obra complexa, 
de acordo com a Diretiva 2001/29/CE (\cite{diretiva-2001-29}), onde o código do jogo é protegido como um "filme interativo".
Este processo judicial decorreu durante vários anos e passou por várias instâncias judiciais, 
até chegar ao Tribunal de Justiça da União Europeia (TJUE).

Antes de mais, é necessário compreender como é considerado um programa de computador/código de jogo. De acordo com o Artigo 1º da Diretiva 2009/24/CE (\cite{diretiva-2009-24}),
um programa de computador é definido como "uma sequência de instruções ou declarações concebidas para serem utilizadas, direta ou indiretamente, numa máquina de 
processamento de dados para efectuar uma função ou tarefa específica ou para obter um resultado específico". Assim, o código do jogo está protegido pela legislação de direitos de autor,
impedindo a sua reprodução ou modificação sem autorização do titular dos direitos, neste caso, a \textit{Sony}. Porém, também é necessário compreender que um jogo está 
armazenado num suporte físico, como um disco UMD (Universal Media Disc) no caso da \textit{PSP}, e que, quando o jogo é executado, o código é carregado na memória volátil 
(\textit{RAM}) do dispositivo, algo que é dinâmico e temporário, ou seja, o código na \textit{RAM} pode ser alterado durante a execução do jogo.


O Advogado-Geral do TJUE, Maciej Szpunar, ao dar as suas conclusões sobre o caso, não conseguiu identificar uma violação dos direitos de autor, como é mostrado no documento 
de conclusões (\cite{conclusoes-advogado}). Ele argumenta, no ponto 57, que os dispositivos de batota não alteram o código do jogo, mas sim a experiência do jogador ao modificar essas variáveis 
presentes na memória volátil (\textit{RAM}) do dispositivo, onde ainda faz uma analogia que facilita a compreensão do caso. Ao imaginar um livro de romance policial, o autor não pode 
impedir que o leitor vá até ao final do livro para descobrir o desfecho da história, mesmo que isso estrague o prazer da leitura e estrague o suspense que o autor tentou criar.
Tudo isto para dizer que, o local que está a ser alterado é um local que está sempre a ser alterado durante a execução do jogo, então a \textit{Sony} nunca conseguirá prever essas alterações com 
certeza absoluta.

No ponto 50, o advogado-geral também destaca que a Diretiva 2001/29/CE não se aplica ao caso, pois a \textit{Sony} apenas baseia-se na condição 
do valor das variáveis gerado por um programa de computador durante a sua execução, constantemente modificado, quer durante essa execução quer 
em cada execução consecutiva, tanto mais que essas modificações não dependem da criação do autor, mas de fatores externos, como os atos dos utilizadores da obra.

Este ainda diz, no ponto 58, que a \textit{Sony} está a confundir-se entre a proteção do direito de autor e a proteção contra a concorrência desleal, já que utiliza o argumento de que 
o programa da \textit{Datel} "se enxerta no programa da \textit{Sony}", porém como este caso está relacionado com direitos de autor, o TJUE não pode considerar este argumento. E acrescenta ainda que,
embora os direitos de autor protejam o código contra a pirataria e contra a contrafação, não impede a utilização da obra de outrem como base para criar algo novo, desde que não haja 
reprodução ou modificação do código original.


Além disso, a \textit{Sony} ainda argumentava que para o dispositivo funcionar, a \textit{Datel} interagia de tal forma com o código do jogo que acabava por criar uma "reprodução" de partes do código original, o que também seria uma violação dos direitos de autor, neste 
caso o Artigo 4º, alínea a) da mesma diretiva, só que isto foi rejeitado pelo TJUE pois, o carregamento do programa na memória \textit{RAM} é um ato necessário para o jogo funcionar.
Após uma análise detalhada do caso, o TJUE considerou apenas o funcionamento técnico do programa, descartando assim a análise consoante a Diretiva 2001/29/CE (\cite{diretiva-2001-29}) sobre obras complexas, como a \textit{Sony} havia solicitado.
No fim, o TJUE concordou com o advogado-geral e entendeu que não havia uma violação dos direitos de propriedade intelectual por parte da Datel. O tribunal considerou que os dispositivos 
de batota não alteram o código do jogo em si, mas alteravam as variáveis do jogo dentro da memória \textit{RAM}. Assim, o tribunal concluiu que a utilização destes dispositivos não constitui uma reprodução ou modificação do programa de computador protegido
pela legislação de direitos de autor, onde isso só seria possível se o código do jogo dentro do disco UMD e/ou no cartão de memória fosse alterado, o que não era o caso.


Sendo assim, a meu ver, a decisão do TJUE foi correta ao concluir que a Datel não infringiu os direitos de propriedade intelectual da Sony, porém é importante destacar que 
se a Sony tivesse apresentado queixa sobre a violação de um direito de concorrência desleal, por exemplo, o resultado poderia ter sido diferente.
A minha opinião coincide com a opinião deste Youtuber alemão, Playground Germany, que fez um vídeo a explicar o caso e a sua opinião sobre o mesmo 
(\cite{youtube_playgroundgermany}). Este defende que a Datel não deveria fornecer dispositivos de batota, pois se um jogador gasta o seu dinheiro num jogo, 
deveria jogar o jogo como foi concebido pelos desenvolvedores, sem recorrer a dispositivos que alteram a experiência de jogo. No entanto, ele destaca que a Sony 
não tinha base legal para processar a Datel com base na legislação de direitos de autor, uma vez que o código do jogo não foi alterado. 

Este youtuber ainda destaca um caso semelhante
que ocorreu entre a \textit{Blizzard Entertainment} e a \textit{MDY Industries}, onde a \textit{Blizzard} processou a \textit{MDY} por criar um software chamado \textit{WoWGlider} que 
permitia aos jogadores automatizar ações no jogo \textit{World of Warcraft}. Neste caso, o tribunal decidiu a favor da Blizzard, já que esta automação fazia uma 
violação dos termos de serviço do jogo, o que é uma questão diferente do caso entre a \textit{Sony} e a \textit{Datel}, apesar de ambos envolverem batotas que faziam alterações 
na memória volátil dos jogos. Além de disso, a \textit{Blizzard} ainda mencionou que o \textit{WoWGlider} poderia prejudicar a experiência de jogo dos outros jogadores, o 
que a faria perder jogadores e, consequentemente, receita. Outra coisa que prejudicou imensamente a \textit{MDY Industries} foi o facto de o tribunal ter considerado que o 
\textit{WoWGlider} violava a Digital Millennium Copyright Act (DMCA) dos Estados Unidos, que proíbe a criação e distribuição de tecnologias que contornem medidas de 
proteção de direitos de autor, algo que no caso da \textit{Datel} não foi mencionado pela \textit{Sony}, já que esta apenas alegava a violação dos direitos de autor.




%---------------------------------------------------------------------------------------------------------------------------

\newpage
\renewcommand{\refname}{Bibliografia} % Para artigos
\renewcommand{\bibname}{Bibliografia} % Para livros e relatórios
\addcontentsline{toc}{section}{Bibliografia} % Adiciona a Bibliografia ao índice
\printbibliography

\end{document}
